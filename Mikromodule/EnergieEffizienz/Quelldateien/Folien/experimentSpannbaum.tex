\documentclass{article}
\usepackage[utf8]{inputenc}
\usepackage{hyperref}
\usepackage{color}
\usepackage{graphicx} 
 \usepackage[ngerman]{babel}



\title{Energieeffizienz beim Coding\\[2mm]Experimente} 
\author{Julia Padberg und Martin Becke}
\date{August 2024}


\begin{document}
\maketitle


\section*{Experiment zu Laufzeit und Energieverbrauch: Spannbaum}

Die folgenden Aufgaben umfassen die Implementierung von  Algorithmen aus dem Bereich der Graphentheorie und  Nutzung der Library \texttt{GraphStream}\footnote{%
   siehe  \href{https://graphstream-project.org/}{https://graphstream-project.org/}}.
	
Es soll der Kruskal-Algorithmus aus GraphStream genutzt werden,  so dass  der minimale Spannbaum   und dessen Kantengewichtssumme berechnet werden.

Die Aufgabe umfasst folgende Teile:

			\begin{enumerate}
			  \item Benutzen Sie den Kruskal-Algorithmus aus Graphstream,
				\item Entwerfen Sie bitte Junit-Tests und			
								
				\item erzeugen Sie dann randomisierte, ungerichtete, gewichtete,  zusammenhängende Multigraphen  
				für die Messungen,
		
				\item in dem Sie  für die Generierung beliebiger, gewichteter Graphen diesen Algorithmus implementieren:
				
			\fbox{\parbox{\linewidth}{\ttfamily
			createWeighted\_MultiGraph(n:int, m:int, min:int, max:int) \\
			\%\%  $1\le n \le m $ und  $min,max$ Intervall für die Kantengewichte\\[2mm]
				ungerichteten, zusammenhängenden Multigraphen $G=(V,E)$\\ initialisieren mit: \\
						 $U=\{v_1, ..., v_n\}$  \hfill  \%\% $n$ ungenutzte Knoten \\	
						 $V=\{\}$  \hfill  \%\% $n$  Knoten \\	
						 $E=\{\}$           \hfill  \%\% Kanten   \\[2mm]							
						select\_randomly $cur \in U$\\
						$U:= U\setminus\{cur\}$ \\
						$V:= V \cup\{cur\}$     \hfill  \%\% $cur$ von $U$ nach $V$ schieben\\[2mm]
						\%\%   je einen  Knoten $nxt \in V, cur \in U$ zufällig	wählen,  \\
						\%\%  Kante dazwischen,  $cur$ von $U$ nach $V$ schieben \\
						\%\%  ergibt  einen Spannbaum\\
			      for $ 1\le i \le n-1\{$  \\ \hspace*{1cm}
						\parbox{.8\linewidth}{
								select\_randomly $nxt\in V$\\ 
								select\_randomly $cur\in U$\\ 								
						    $U:= U\setminus\{cur\}$ \\
						    $V:= V \cup\{cur\}$\\
								select\_randomly $min \le weigth \le max$\\ 
								$E:=E \cup edge(cur,nxt, weight)$\\
								$cur:=nxt$\\
									$\}$
									}
								\\[2mm]
								\%\%  restliche Kanten einfügen\\[2mm]
								for $ n < i \le m\{$  \\ \hspace*{1cm}
						\parbox{.8\linewidth}{
								select\_randomly $cur\in V$\\
								select\_randomly $nxt\in V$\\
								select\_randomly $min \le weigth \le max$\\ 
								$E:=E \cup edge(cur,nxt, weight)$\\
									$\}$
									}
								\\
						return $G$
				}}				
				
\newpage

       \item Laufzeit- und Energieverbrauchsmessung\\
			    Für die Messung erzeugen Sie bitte  3  Graphen\\ \hspace*{1cm} 
					mit 100 Knoten \& 3.000 Kanten, \\  \hspace*{1cm} 
					mit 1.000 Knoten \& 400.000 Kanten und \\ \hspace*{1cm} 
					mit 10.000  Knoten \& 2.000.000 Kanten.\\
					Dann lassen Sie Ihre Algorithmen mehrfach darauf laufen.\\
					
					Für jeden der drei Graphen erstellen Sie bitte  eine Tabelle, z.b CVS-Datei, 
					in der die Kantenanzahl, Ihre Hardware und Ihr Messwerkzeug geschrieben werden, in etwa so:
					
					\begin{tabular}{|c|c|}
						\hline
						\multicolumn{2}{|l|}{Hardware, IDE, Java-Version, } \\
						\multicolumn{2}{|l|}{ Tool für die Messung, Knotenanzahl, Kantenanzahl} \\ \hline \hline
						Laufzeit & Energie \\ \hline
						 ... &  ...\\ \hline
					\end{tabular}
			
					Danach werden 13 Durchläufe, von denen nur die letzten 10 ausgewertet werden, gestartet und 
					dann für jeden Durchlauf (ab dem 4.) die Zeit in Millisekunden und der Energieverbrauch in Joule 
					(oder was auch immer das Messwerkzeug raus gibt)
					in die  CSV-Datei geschrieben. 
					
					
					\item Statistische Analyse in Excel\\
					      Wählen Sie die Zelle aus, in der Sie den Korrelationskoeffizienten berechnen möchten.
								Verwenden Sie diese Formel:  								
								\begin{center}
								\texttt{=KORREL(A4:A14, B4:B13)}
								\end{center}
								wobei die Spalten durch die Felder 
								eingeben werden.  Nach Eingabe der Formel drücken Sie Enter. 
								Die Zelle wird nun den Korrelationskoeffizienten anzeigen.
Die Werte des Pearson-Korrelationskoeffizienten liegen zwischen -1 und 1. Ein Wert von 1 zeigt eine perfekte positive lineare Beziehung an, -1 eine perfekte negative lineare Beziehung, und 0 bedeutet keine lineare Beziehung.
					\end{enumerate}

\end{document}