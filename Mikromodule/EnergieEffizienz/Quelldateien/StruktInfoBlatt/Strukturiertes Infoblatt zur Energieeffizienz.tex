\documentclass[utf8]{article}
\usepackage[utf8]{inputenc}
\usepackage[ngerman]{babel}

\begin{document}

%\title{Computer Science for Future\\Nachhaltigkeit und Klimaschutz in den Informatikstudiengängen der HAW Hamburg

\title{Was ist Energie-Effizienz in der Programmierung? Dasselbe wie Laufzeitperformanz?}

\author{Haron Nazari
        \and Julia Padberg\footnote{email: ComputerScience4Future@haw-hamburg.de }
				\\[3mm]
				\large Hochschule für Angewandte Wissenschaften  Hamburg}

%

\maketitle
%

\section{Einleitung}
Energieeffizienz in der Programmierung bezieht sich auf die Fähigkeit eines Programms, eine Aufgabe auszuführen und dabei so wenig Energie wie möglich zu verbrauchen. Energieeffizienz ist ein wichtiger Aspekt bei der Softwareentwicklung, insbesondere bei mobilen und batteriebetriebenen Geräten, bei denen der Energieverbrauch ein entscheidender Faktor für die Lebensdauer der Gerätebatterie ist. Energieeffizienz bei der Programmierung kann durch die Erstellung von effizienterem Code erreicht werden, der weniger Arbeit verrichtet, durch eine einfache Parallelisierung der Arbeit, damit die Kerne schneller in den Ruhezustand gehen können, oder durch eine Optimierung der Speichernutzung.
Die Energieeffizienz eines Programms ist nicht zwangsläufig direkt an die Laufzeit gekoppelt, obwohl es eine gewisse Wechselwirkung zwischen beiden geben kann. Die Laufzeit eines Programms bezieht sich auf die Zeit, die benötigt wird, um einen bestimmten Prozess oder eine Aufgabe abzuschließen. Die Energieeffizienz hingegen bezieht sich auf die Menge an Energie, die für die Ausführung des Programms benötigt wird, um diese Aufgabe zu erledigen. Die Beziehung zwischen Energieeffizienz und Ausführungszeit jedoch noch unklar. Es wird zwar allgemein angenommen, dass der Energieverbrauch mit der Ausführungszeit zusammenhängt, aber es gibt in der Literatur unterschiedliche Ergebnisse: \cite{yuki_Folkore_2014} behaupten, dass der Zusammenhang zwischen Energieeffizienz und Performance gegeben ist. In \cite{pereira_energy_2017}  wird konstatiert, dass  eine Abhängigkeit nicht notwendig vorliegen muss. \cite{trefethen_energy-aware_2013}  zeigten, dass die Wahl des Compilers, die Frequenz der CPU und die Anzahl der Threads einen erheblichen Einfluss auf den Energieverbrauch bei gleicher Leistung haben. \cite{georgiou_analyzing_2017} zeigen für unterschiedliche Programmiersprachen, dass der Energieverbrauch in allen Fällen (außer bei Rust)  direkt von der Laufzeitleistung abhängt.


\section{Forschungsfrage}
\textbf{Lassen sich statistisch relevante Unterschiede zwischen Laufzeit und Energieverbrauch finden?}
Es wird  unterschiedlicher Code (verschiedene Implementierungen von vorgebenden Algorithmen)  für unterschiedliche große Eingaben in mehrfache Durchläufe mit variierenden Werkzeugen zur Energiemessung auf unterschiedlicher Hardware  laufen gelassen und der Energieverbrauch und die Zeit gemessen. Diese Reihen werden dann auf Korrelation getestet, da es ich bei Laufzeit und Energieverbrauch um metrische Daten handelt, wird der Pearsontest dafür genutzt. 

\section{Technische Umsetzung und  Methodik}
Unterschiedliche Software Werkzeuge stehen dafür zur Verfügung:
\begin{itemize}
	\item Power Gadget von Intel
  \item JoularJX
  \item und andere siehe auch „Tools für Software Strommessung“
\end{itemize}
Um den tatsächlichen Energieverbrauch von Programmiercode zu messen, ist eine kontrollierte Ausführung erforderlich. Um vergleichbare Ergebnisse zu erhalten und Zusammenhänge zu identifizieren, sollte die Messung und Ausführung des Codes unter reproduzierbaren Bedingungen erfolgen. Unit-Tests bieten eine automatisierte Möglichkeit, alle im Test definierten Aufrufe und Instanziierungen durchzuführen, und sind leicht erweiterbar, was den Aufwand für die Messungen begrenzt. Die Aufgaben benötigen einen Unit-Test, damit  alle Messungen in einem einzigen Durchlauf ausgeführt werden können.


\section{Aufgabe}
Für ein oder zwei zu implementierende Algorithmen  wird eine Laufzeit- und Energieverbrauchs-messung durchgeführt. In den vorgegebenen Unit-Tests werden Eingaben für die Algorithmen randomisiert in unterschiedlicher Größe jeweils 15-mal erzeugt und 10 mal durchlaufen. Dabei werden in eine CSV-Datei zunächst Hardware und  Messungswerkzeug geschrieben. Danach werden 15 Durchläufe, von denen nur die letzten 10 ausgewertet werden, gestartet und dann für jeden Durchlauf die Zeit in Millisekunden und der Energieverbrauch in Joule in eine CSV-Datei geschrieben. 


\section{Was hat das mit mir zu tun?}
Ich möchte wissen,  wie ich den Energieverbrauch meiner Implementierung messen kann. Und ich möchte verstehen, wie Energieverbrauch und Laufzeitperformanz zusammenhängen.


\bibliography{sI_EnergieEff}
\bibliographystyle{plain}


\end{document}