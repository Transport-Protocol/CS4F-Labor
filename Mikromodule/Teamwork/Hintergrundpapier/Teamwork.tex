\documentclass[a4paper]{article}
\usepackage[utf8]{inputenc}
\usepackage[T1]{fontenc}
\usepackage[ngerman]{babel}
\usepackage{xcolor}

\usepackage{hyperref}

\newcommand{\komm}[1]{{ \color{blue}}}%\textbf{Kommentar}: #1\\~}}


\title{TEAM - die Bedeutung von Teilnahme, Effizienz, Ambition und Motivation in Projekten}
\author{Nachhaltigkeitslabor Department Informatik (HAW Hamburg)}
\date{2024}

\begin{document}

\maketitle

\section{Einführung}

%Teamfähigkeit ist eine essentielle Stärke in vielen Arbeitsbereichen. Auch in der Informatik hat sich mit der Zeit der Stereotyp des "`lone wolf"' Informatikers gelegt. Stattdessen wird nun um Teamwork und soziale Kompetenz geworben. Doch in MINT-Fächern, besonders in der neuen Aufmachung in der Branche, kommt die Lehre solcher Kompetenzen zu kurz. So haben Absolventen zwar Kontakt mit Gruppenarbeiten und Projekten, weisen jedoch nicht die gefragten Eigenschaften auf. Die Lehre dieser muss somit im Lehrplan aufgenommen und Themen wie Kommunikation und Konfliktmanagement konkret behandelt werden.

Teamwork ist ein entscheidender Bestandteil vieler Bildungs- und Arbeitsumgebungen, kann jedoch Herausforderungen mit sich bringen. Auch in der Informatik werden immer mehr Teamfähigkeit sowie soziale Kompetenzen angefordert, was sich mit dem bisher bekannten Stereotypen des Informatikers beißt. So ist das Bild der kommunikationsmeidenden und alleine arbeitenden IT nicht mehr aktuell und wird aktiv durch gegensätzliche Anforderungen am Arbeitsplatz verändert.\komm{welche Eigenschaften?}\\
Die Herausforderungen bestehen jedoch auch schon vor dem Schritt in die Arbeitswelt. Besonders in MINT-Fächern wird die Lehre dieser Kompetenzen außer Acht gelassen, wodurch Absolventen, auch wenn sie an Gruppenarbeiten teilgenommen haben, Schwierigkeiten damit haben, sich in neue Teams mit fremden Menschen zu integrieren und effektiv zu arbeiten \cite{teamwork_teaching_engineering_china}. 
\komm{missglückte Teamarbeit äußert sich in diskriminierendem Verhalten? Quellen?}
Im folgenden werden verschiedene Studien und Forschungsergebnisse zum Thema Teamarbeit und Kommunikation vorgestellt sowie Ideen und Möglichkeiten für die Lehre präsentiert.\\

\section{Essentielle Probleme}
Diverse Studien beschäftigen sich bereits mit den zahlreichen Problemen, welche unabhängig vom Umfeld in Teams auftauchen können. So zeigt \cite{balanced_teamwork_scrum} unter anderem eine Ungleichheit in der Aufteilung der Arbeitslast. Teammitglieder weisen dabei Unterschiede im angewandten Arbeitsaufwand auf, was auch unterschiedliche Ausprägungen von Verantwortungsbewusstsein und Disziplin vermuten lässt. Eine geeignete Teamaufstellung zu finden und zu entscheiden, wie und nach welchem Prinzip die Aufgaben verteilt werden, sind wichtige Aspekte, die stark zur korrekten Aufteilung der Arbeitslast beitragen können \cite{teamworkstrat_firstsem_engineering}.
\\\\
Ein weiteres Problem sind Kommunikationsbarrieren aufgrund von kulturellen und anderweitigen Unterschieden zwischen Teammitgliedern \cite{Multicultural_miscommunication}. Bei Befragungen konnte festgestellt werden, dass Gruppenmitglieder (besonders Minderheiten) sich in einigen Fällen ausgegrenzt fühlen und Konflikte nicht behandelt, sondern unterdrückt werden, um das Teamwork "`effizient"' und utilitaristisch zu halten \cite{inclusivity_teamwork_southafrica}. Es ist wichtig, die Kommunikationsfähigkeit zu verbessern, um innerhalb von diversen Gruppen mögliche Missverständnisse und Konflikte aufzuheben und vorzubeugen. Dinge wie Verantwortungen, Erwartungen, Aufgabenaufteilung und Rücksicht kommen in Mischgruppen teilweise zu kurz \cite{gender_differences}, was nicht nur die Gruppenleistung beeinträchtigt, sondern auch dazu führt, dass Mitglieder Gruppenarbeiten im allgemeinen negativ auffassen und keine Mühe mehr für diese aufwenden \cite{teamwork_teaching_engineering_china}.
\\\\
Eine zentrale Problematik, die zu alldem beiträgt, ist die Lehre. Es wurde festgestellt, dass eine Veränderung in der Lehre, also ein Aufbau neuer Lehrmodelle und die Durchsetzung von Teamarbeit in den Modulen die Leistung und Motivation sowie Lernfähigkeit von Studierenden verbessern kann \cite{blended_teaching_innovationability}. Dieser Neuaufbau enthält einen studierendenfokussierten Ansatz, in welchem die klassischen Vorlesungen durch aktivere Einheiten/interaktive Lehre ersetzt werden, um nicht die Vermittlung, sondern das Verständnis des Materials zu fördern.
Hilfreich ist in dem Sinne die Anwendung von kleineren Gruppenprojekten, welche bereits früh im Studium stattfinden und die Entwicklung der Teamfähigkeit fokussieren, statt die technischen Inhalte des erarbeiteten Themas. Diese Projekte werden also gezielt als Mittel genutzt, um die Arbeit und Kommunikation in verschiedenen Gruppendynamiken zu erproben \cite{4Cs_teamwork_china}.
\\
Teamfähigkeit sollte jedoch nicht nur praktisch durch die Projekteinbindung in die Lehrmodule, sondern auch inhaltlich vermittelt werden. Häufig wird dies nicht berücksichtigt, da der Mehrwert eines neuen Moduls oder von neuen Inhalten im Gegensatz zu den Kosten und Mühen, die eine solche Änderung mit sich bringt, als ungenügend gesehen wird. Fakultäten, die nicht bereits in diese Inhalte investieren, sind hiervon betroffen und sparen sich lieber Zeit und Geld, da der Nutzen nicht eingeschätzt werden kann \cite{communication_teamwork_engineering}.
\\\\
Zu letzt gibt es auch den persönlichen Aspekt. Studierende werden bei aufgetragenen Gruppenarbeiten mit der Unsicherheit konfrontiert, ob die Persönlichkeiten und Motivationen innerhalb des Teams zueinander passen. Es besteht die Angst, dass die jeweiligen Mitglieder nicht dieselbe Motivation, Idee, Kommunikationsfähigkeit oder die passenden Stärken aufweisen \cite{teamworkstrat_firstsem_engineering}. Der Fakt, dass solche Unsicherheiten durch geeignete Kommunikation und Rücksicht behoben werden können, muss allerdings vermittelt werden, sodass alle Gruppenmitglieder zumindest aus Rücksicht zu ihren Teamkollegen zu einer guten Zusammenarbeit beitragen.\\

\section{Vorschläge und Versuche für die Verbesserung der Teamfähigkeiten}

Verschiedene Ansätze wurden getestet, um die Kommunikations- und Teamfähigkeit von Studierenden zu verbessern.
So beginnen die Verbesserungsvorschläge für gelungenes Teamwork bereits bei der Gruppenaufstellung \cite{teamworkstrat_firstsem_engineering}. Eine beschränkt interessenbasierte Gruppenwahl mithilfe von vorher aufbereiteten Themenpools zeigt dabei Potenzial, Meinungsdifferenzen zwischen Mitgliedern einer heterogenen Gruppe zu minimieren und auch die Motivation für die Aufgabenbearbeitung zu maximieren, da jedes Mitglied zumindest in ein für sie (im Vergleich) interessanteres Themengebiet eingeteilt werden würde.\\\\
\komm{viel Gruppenarbeit im Studium it jedoch nicht in dem Maße iteressengeleitet, mit welchen techniken kann dann die Zusammenarbeit verbessert werden}% Jetzt agile Methoden und Kommunikationskanäle für Koordinierung
Für die Aufgabenbearbeitung in den gebildeten Gruppen werden vor allem agile Methoden vorgeschlagen, da die Teams sich so selbst organisieren müssen und die einzelnen Mitglieder aufgrund der Rollenverteilung ihre Verantwortungen besser identifizieren können \cite{balanced_teamwork_scrum} \cite{teamwork_teaching_engineering_china}. Durch die agile Projektplanung und Rollenverteilung wird außerdem die ungleiche Aufgabenlast aufgelöst, da jeder sich schrittweise um einen eigenen festgelegten Aufgabenbereich zu kümmern hat und die Aufteilung unter Absprache erfolgt. Für die Koordination und Planung sind Meetings hilfreich, es sollten aber auch Kommunikationskanäle gewählt werden, die außerhalb der "`offline"' Meetings verwendet werden \cite{teamwork_teaching_engineering_china}. \\\\
% Kommunikationsfähigkeit und Diversität
Um diese teaminterne Struktur zu stärken und die Kommunikationskanäle sowie Meetings effektiv zu halten, muss außerdem die allgemeine Kommunikationsfähigkeit weitergebildet werden. Durch rücksichtsvollen Ideenaustausch und Diskussionen kann unter anderem das Gruppenbewusstsein verbessert werden \cite{4Cs_teamwork_china}. Teil einer positiven Erfahrung in Gruppenarbeiten ist außerdem die gemeinsame Ideenentwicklung und Förderung der Kreativität durch einen offenen Austausch \cite{inclusivity_teamwork_southafrica} \cite{4Cs_teamwork_china}. Ein weiterer positiver Effekt ist ein inklusiver Umgang in Teams mit diversen Hintergründen, sodass jedes Teammitglied unabhängig von Herkunft, Sexualität etc. gleich behandelt wird und eine positive Teamwork-Erfahrung sammeln kann. \\\\
Kommunikationsfähigkeit mit allen ihren Einzelteilen ist jedoch nicht nur teamintern wichtig, sondern auch im Austausch mit anderen Gruppen. Daher wird auch die Lehre der effizienten Kommunikation und des allgemeinen verbalen Ausdrucks (z. B. Präsentation des Zwischenstands in der Projektbearbeitung) vorgeschlagen, um die Redefreiheit und Selbstsicherheit zu stärken und die Wichtigkeit der Teamrepräsentation aufzugreifen \cite{lifeskills_engineers}.\\\\
% Neuaufbau von Lehrmodulen
Die genannten Punkte zur Verbesserung der Teamfähigkeiten der Studierenden müssen theoretisch wie auch praktisch erarbeitet werden. 
\komm{Mikro-Module?}
Dafür sollten Fakultäten in die Aufnahme neuer Module oder die Umstellung der bestehenden investieren. Eine mögliche Umstellung zeigt \cite{blended_teaching_innovationability}, wo ein kleiner gefasstes "`inverted classroom"' Konzept genutzt wurde, in welchem zu kurzen inhaltlichen Videos Problemstellungen behandelt werden sollten. Eine dreistufige Praktikumseinheit, in welcher jedes Thema auf einfacher, dann komplexer und zuletzt experimenteller Stufe praktisch in Gruppen erarbeitet werden sollte, förderte die Motivation und Lernfähigkeit der Studierenden.\\
Ähnlich dazu können auch kleinere Projekte (zu Beginn des Studiums) eingeführt werden, in welchen die erarbeiteten Inhalte an zweiter Stelle stehen und der Fokus der Lehre und Projektwertung auf der aufgewiesenen Teamfähigkeit während der Bearbeitung sowie dem korrekten Planen und Erarbeiten eines Projekts liegen \cite{teamworkstrat_firstsem_engineering} \cite{4Cs_teamwork_china}.\\
Vorteilhaft wäre zudem ein begleitender theoretischer Kurs für Kommunikation und Konfliktmanagement, damit Studierende das Handwerkszeug für eine gelungene Gruppenarbeit erhalten und direkt anwenden können \cite{teamworkstrat_firstsem_engineering}.
\komm{ein Mikro-Modul umfasst maximal 180 min, also ein theoretischer Einstieg wäre vielleicht gut oder ein konkretes Szenarion}

%Die Ergebnisse zeigen, dass agile Methoden wie Scrum dabei helfen, die Aufgabenteilung gerechter und logischer vorzunehmen \cite{balanced_teamwork_scrum}. Die Versuchsgruppen konnten ihre Verantwortungen identifizieren und annehmen, indem sie eine Rollenverteilung vornahmen und für Absprachen einen Kommunikationskanal errichteten.\\\\
%Hilfreich schien auch die Nutzung des "`inverted classroom"' Konzepts in kleineren Ausmaßen \cite{blended_teaching_innovationability}, wobei zu kurzen inhaltlichen Videos Problemstellungen behandelt werden sollten. Eine dreistufige Praktikumseinheit, in welcher jedes Thema auf einfacher, komplexer und zuletzt experimenteller Stufe praktisch in Gruppen erarbeitet werden sollte, förderte die Motivation und Lernfähigkeit der Studierenden.\\\\
%Für die Kernprobleme lohnt es sich auch, die Teamaufstellung zu betrachten. So ist es hilfreich, verschiedene Methoden zu testen, über welche die Gruppen aufgestellt werden. Eine interessenbasierte Gruppenaufstellung zeigte sich als nützlich durch die gemeinsame Zielsetzung und zugrunde liegende Motivation \cite{teamworkstrat_firstsem_engineering}. Es wurde jedoch auch gezeigt, dass selbst bei gut aufgestellten Gruppen Konfliktpotential herrscht. Ein begleitender theoretischer Kurs für Kommunikation und Konfliktmanagement wie in der Studie wäre also von Vorteil.\\\\
%Um in bestehenden Gruppen diese Konflikte zu minimieren, wurde die gemeinsame Ideenentwicklung und Koordination von Meetings von Studierenden als wichtig eingestuft \cite{inclusivity_teamwork_southafrica}. Für eine Art Teamtraining werden auch Peer-Assessments und Rollenspiele vorgeschlagen. Um weiterhin das Gruppenbewusstsein und das Teamgefühl, besonders in diversen Gruppen, zu stärken, ist außerdem die nötige Rücksicht und Inklusivität gefragt \cite{4Cs_teamwork_china}. Ein gutes Verhältnis in der Gruppe soll dann nicht nur die Ausgrenzung anderer vermeiden, sondern auch die Kreativität und den Ideenaustausch fördern.\\\\
%Rücksicht und Inklusivität sind Teil der Kommunikationsfähigkeit, welche innerhalb einer Gruppe zu tragen kommt. Sowie in Meetings, als auch in gemeinsam erwählten Kommunikationskanälen sollte ein Austausch gepflegt werden, um weiterhin den Teamzusammenhalt und die inhaltiche Arbeit zu fördern \cite{} Weiterhin ist diese Eigenschaft aber auch nach außen hin wichtig in dem Sinne, dass die sprachliche Kompetenz ausgebaut werden muss, um das Team repräsentieren zu können und Beiträge mit anderen Teams austauschen zu können \cite{lifeskills_engineers}.

\komm{was sit mit dem auskommentiertem?}
%\section{Möglichkeiten für uns}

%Die Vorschläge können an der HAW vorerst in einem kleineren Umfang angepasst und getestet werden. Die Idee ist dabei, ein kleines Projekt zu entwickeln, welches von Erstsemester-Studis in der Orientierungswoche bearbeitet werden soll.\\Die Projektthemen sollen aus einem Themenpool frei wählbar sein, sodass die Gruppeneinteilung möglichst interessenbasiert erfolgt. Um wichtige Aspekte der Teamarbeit in kurzer Zeit an die Studierenden heranzuführen, und die Organisation als Gruppe zusätzlich zu stützen, wird der Themenpool mit Teamwork-Themen wie z.B. "'effiziente Kommunikation"', "'Rollenverteilung"', "'Risikomanagement"' etc. gefüllt. \komm{die Themen besser zum Thema Teamarbeit, damit sich die Oragnisation als Team durch die Beschäftigung mit dem Thema gestützt wird.}\\
%Die wählbaren Themen sollen Raum für Austausch und Kreativität bieten und noch nicht stark in die Materie des Studiums einsteigen, sodass die Studierenden nicht unter Zeit- oder Leistungsdruck geraten, sondern Spaß haben oder sogar einen Vorgeschmack auf die Studieninhalte bekommen.\\\\ Daneben sollte eine Art Seminar angeboten werden, in welchem kurz der Projektablauf sowie eine Übersicht über Kommunikations- und Projektmanagementmethoden vorgestellt wird. In diesem Fall kann ein allgemeines Mikro-Modul entwickelt werden, welches in der Orientierungseinheit und eventuell auch einem Pflichtmodul Anwendung findet. 
%Ziel ist, dass die Teilnehmenden eine gute erste Erfahrung innerhalb einer Gruppenarbeit erhalten und somit die nötigen Verhaltensweisen und eine positive Einstellung in die nachfolgenden Gruppenarbeiten im Studium sowie in die Arbeitswelt mitnehmen. Es bietet auch die Möglichkeit Ausgrenzungen zu vermeiden, indem der rücksichtsvolle Umgang miteinander und die Identifikation sowie Vorbeugung von diversen Konfliktpotenzialen erlernt wird. Durch die korrekte Vorgehensweise in der Arbeit miteinander kann sichergestellt werden, dass auch Studierende verschiedenster Hintergründe ihre Beiträge ohne Hindernisse leisten und sich durch gesunden Austausch weiterentwickeln können. So kann ein Baustein geboten werden, durch welchen Interessenten Spaß am Studium haben und nicht durch Ausschluss aus der Studierendengemeinschaft den Abschluss aufgeben.
\komm{das Thema Gerechtigkeit \& Teilhabe vertiefen}


% interessenbasiert problematisch
% trennen: wissenschaftlicher hintergrund und möglichkeiten und projektentwicklung haw, lerneffekt etc
% Podcast

\newpage
\bibliographystyle{plain}
\bibliography{ref.bib}

\end{document}
