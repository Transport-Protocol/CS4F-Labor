\documentclass[a4paper]{article}
\usepackage[utf8]{inputenc}
\usepackage[T1]{fontenc}
\usepackage[ngerman]{babel}
\usepackage{xcolor}

\usepackage{hyperref}


\title{Teamwork an der HAW - Möglichkeiten Für Uns}
\author{Nachhaltigkeitslabor Department Informatik (HAW Hamburg)}
\date{May 2024}

\begin{document}

\maketitle

%\section{Möglichkeiten für uns}

Die Vorschläge können an der HAW vorerst in einem kleineren Umfang angepasst und getestet werden. Die Idee ist dabei, ein kleines "Projekt" zu entwickeln, welches von Erstsemester-Studierenden in der Orientierungswoche bearbeitet werden soll.\\\\Die Themen können entweder aus einem Themenpool frei wählbar sein, sodass die Gruppeneinteilung möglichst interessenbasiert erfolgt, oder durch die OE-Leiter zugewiesen werden, um die Zufälligkeit der Gruppeneinteilung zu gewährleisten. Da wichtige Aspekte der Teamarbeit in kurzer Zeit an die Studierenden herangeführt werden sollen und die Organisation als Gruppe zu Beginn zusätzlich gestützt werden muss, wird der Themenpool mit Teamwork-Themen wie z.B. "`effiziente Kommunikation"', "`Rollenverteilung"', "`Risikomanagement"' etc. gefüllt. %\komm{die Themen besser zum Thema Teamarbeit, damit sich die Oragnisation als Team durch die Beschäftigung mit dem Thema gestützt wird.}\\
Die wählbaren Themen sollen Raum für Austausch und Kreativität bieten und noch nicht stark in die Materie des Studiums einsteigen, sodass die Studierenden nicht unter Zeit- oder Leistungsdruck geraten, sondern Spaß haben oder sogar einen Vorgeschmack auf die Studieninhalte bekommen.\\\\ Für die Themen werden dann Arbeitsblätter ausgeteilt, welche kurze Infotexte zu den Themen enthalten. Innerhalb der Gruppen werden dann vorgegebene Fragestellungen bearbeitet und dann im Plenum vorgestellt.\\\\ Daneben kann auch eine Art Seminar angeboten werden, in welchem kurz der Projektablauf sowie eine Übersicht über Kommunikations- und Projektmanagementmethoden vorgestellt wird. In diesem Fall sollte ein Mikro-Modul entwickelt werden, welches in der Orientierungseinheit und eventuell auch einem Pflichtmodul wie Software Engineering~1 Anwendung findet. Das Micro-Modul selbst soll dabei zwei Ausprägungen haben und sich im Falle von Software Engineering~1 auch mehr mit Themen wie Diskriminierung beschäftigen, da dort mehr Zeit und Studien- und Teamworkerfahrung geboten wird.\\\\
Ziel des OE-Projekts ist, dass die Teilnehmenden eine gute erste Erfahrung innerhalb einer Gruppenarbeit erhalten und somit die nötigen Verhaltensweisen und eine positive Einstellung in die nachfolgenden Gruppenarbeiten im Studium sowie in die Arbeitswelt mitnehmen. Es bietet auch die Möglichkeit Ausgrenzungen zu vermeiden, indem der rücksichtsvolle Umgang miteinander und die Identifikation sowie Vorbeugung von diversen Konfliktpotenzialen erlernt wird. Durch die korrekte Vorgehensweise in der Arbeit miteinander kann sichergestellt werden, dass auch Studierende verschiedenster Hintergründe ihre Beiträge ohne Hindernisse leisten und sich durch gesunden Austausch weiterentwickeln können. So kann ein Baustein geboten werden, durch welchen Interessenten Spaß am Studium haben und nicht durch Ausschluss aus der Studierendengemeinschaft den Abschluss aufgeben.

\end{document}
