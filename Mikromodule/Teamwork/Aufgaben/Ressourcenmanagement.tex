\documentclass[a4paper]{article}
\usepackage[utf8]{inputenc}
\usepackage[T1]{fontenc}
\usepackage[ngerman]{babel}

\title{\vspace{-2cm}Ressourcenmanagement}
\author{CS4F}
\date{April 2024}

\begin{document}

\maketitle


Ressourcenmanagement ist eine entscheidende Teilaufgabe des Projektmanagements\footnote{https://asana.com/de/resources/resource-management-plan}
. Der Erfolg eines Projekts hängt nicht nur von der Motivation und den Fähigkeiten der Teammitglieder ab, sondern auch von der Verfügbarkeit und der effektiven Nutzung der Ressourcen beispielsweise durch die Nutzung von Ressourcenmanagement-Plänen. Eine klare Übersicht über die Ressourcen, Aufgaben und den Projektzeitplan hilft, potenzielle Probleme frühzeitig zu erkennen und ein erfolgreiches Projektmanagement sicherzustellen.
In dem Sinne gehen Ressourcen- und Risikomanagement Hand-in-Hand: Wenn das Ressourcenmanagement versagt, entsteht ein neues Risiko für das Team und Projekt. Andersherum können unvorhergesehene oder unbehandelte Risiken dazu führen, dass eine Umplanung der Ressourcennutzung nötig ist. \\

Ressourcenmanagement umfasst zum Beispiel die Lasteneinteilung, Kostenabwägung und Nutzung von Personal und ihren Fähigkeiten. Dazu müssen einem die Stärken, Schwächen sowie wichtige Lebensumstände jedes Teammitglieds bekannt sein. Anhand dieser Kenntnisse können die einzelnen Teammitglieder dann auf den selben Wissensstand im Sinne der Inhalte oder des Kommunikationsverhaltens. Wichtig ist deshalb auch die Puffereinplanung in jedem denkbaren Aspekt, sei es Zeit, Geld oder anderes, um ein realistisches Projektziel zu setzen und die Anforderungen unter diversen Umständen erfüllbar zu machen.\\\\\\

\begin{flushleft}

\textbf{Aufgabe:}\newline \newline
Szenario: Sie arbeiten in einer Gruppe aus 6 Leuten. 2 dieser Leute haben Programmieren im Laufe eines Ingenieur-Studiums in Deutschland und 2 im Ausland gelernt. Eine Person hat Familie, zwei weitere haben gesundheitliche Einschränkungen (Person A: Angststörung, Person B: eine chronische Lungenerkrankung). 3 der Teammitglieder haben einen Migrationshintergrund, alle sind auf der Stufe B2, einer absolviert aktiv einen weiteren Deutschkurs. Alle Teammitglieder haben unterschiedliche kulturelle Hintergründe oder wurden unter verschiedensten Umständen großgezogen und haben somit auch unterschiedliche Verständnisweisen was Offenheit, Arbeitsethik und Kommunikationsstil betrifft. \newpage\textbf{Person A:} verschweigt gerne, dass sie Schwierigkeiten hat und bittet ungerne um Feedback. \newline \textbf{Person B:} Tut sich schwer darin, sich selbst einzuschätzen und übernimmt somit zu viel oder zu wenig der Arbeitslast, wodurch viele Puffer/Umplanungen nötig sind. \newline \textbf{Person C:} Kann Deutsch, drückt sich jedoch teilweise undeutlich aus durch den technischen Jargon, der in die Sätze eingebaut werden muss. \newline \textbf{Person D:} Hat starke Unterschiede im Verständnis der Arbeitsethik und beschäftigt sich mit dem Minimum, stellt somit häufig eine unterbelastete Ressource dar. \newline \textbf{Person E:} Ist ein Platzhirsch, welcher andere Lösungswege nicht anerkennt und andere Beiträge unterbricht. \newline \textbf{Person F:} Hat Familie, darunter 2 Kleinkinder.\newline \newline

\textbf{Aufgabe 1.}\newline Was für Probleme müssen beim Ressourcenmanagement beachtet werden? Welche konkreten Umstände aus dem oben abgegebenen Szenario müssen besonders berücksichtigt werden und könnten zu einer Umplanung führen? Tauschen Sie sich in Ihrer Gruppe aus.\newline \newline
\textbf{Aufgabe 2.} \newline Suchen Sie sich zwei der Probleme aus und spielen Sie diese aus. Was wären die Folgen, wenn man diese Probleme nicht beachtet?\newline

\textbf{Aufgabe 3.} \newline Die meisten Probleme müssen durch vorausschauendes Planen vorgebeugt werden. Was sind mögliche Gegenmaßnahmen/Vorgehensweisen um das Projekt weiterzuführen? Wird ein Puffer benötigt? Wenn ja, für welches Problem? Was sind mögliche Vorgehensweisen, wenn auch der Puffer nicht ausreicht? \newline
\end{flushleft}

\end{document}
