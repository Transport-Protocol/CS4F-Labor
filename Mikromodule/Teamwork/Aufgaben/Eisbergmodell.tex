\documentclass[a4paper]{article}
\usepackage[utf8]{inputenc}
\usepackage[T1]{fontenc}
\usepackage[ngerman]{babel}

\title{\vspace{-2cm}Eisbergmodell}
\author{CS4F}
\date{May 2024}

\begin{document}

\maketitle


Nach Freud gibt es zwei Kommunikationsebenen\footnote{https://studyflix.de/biologie/eisbergmodell-2693}, auf denen man sich ausdrücken oder die Kommunikationspartner verstehen kann. Dabei ist die Spitze des Eisbergs, der kleinere Anteil der Kommunikation, die sogenannte „Sachebene“. Hier sind sachliche Informationen, die verbal und bewusst ausgedrückt werden, aufzufinden. Auf der „Beziehungsebene“ findet wiederum die unbewusste Kommunikation statt, also die, die nicht unbedingt durch verbalen Ausdruck erkennbar ist. Hier sind Dinge wie Mimik und Gestik oder auch der Tonfall im verbalen Ausdruck wichtig. \\\\

Nach Freud gibt es zusätzlich noch das Drei-Instanzen-Modell, welches mit dem Eisbergmodell kombiniert wird. Die drei Instanzen sind das „Ich“, „Über-Ich“ und das „Es“.
„Ich“: Realitätsprinzip, übernimmt das Handeln und die Kommunikation. Somit ist diese Instanz weit oben im Eisbergmodell und hat den kleinsten Anteil in der Beziehungsebene.
„Über-Ich“: Moralitätsprinzip, richtet sich nach Werten und Moral. Diese Instanz ist wie das „Ich“ sowohl auf der Sachebene als auch auf der Beziehungsebene vertreten, wobei der größere Anteil die Beziehungsebene darstellt.
„Es“: Lustprinzip, folgt den Bedürfnissen und Wünschen. Die letzte Instanz befindet sich nur auf der Beziehungsebene. Die Bedürfnisse und Wünsche sind somit non-verbal und unbewusst vertreten.\\\\

Konflikte können auf beiden Ebenen auftauchen. Sachliche Konflikte werden durch z.B. faktische Missverständnisse oder fehlendes Wissen ausgelöst. Konflikte auf der Beziehungsebene sind dagegen tiefgreifender, da hier Werte, Moral und Erwartungen mitspielen und somit den Konflikt auch in die Sachebene tragen. Die Kommunikation ist dann auf beiden Ebenen betroffen. \\\\


\begin{flushleft}
%\textbf{Aufgabe 1.}\newline Wie können solche Konflikte entstehen? \newline \newline
%\textbf{Aufgabe 2.}\newline Wie können Konfliktauslöser auf der Sachebene in Teamarbeiten aussehen? Nennen Sie Beispiele. Überlegen Sie, wie die drei Instanzen in diesen Fällen aussehen. \newline \newline
%\textbf{Aufgabe 3.}\newline	Wie können Konfliktauslöser auf der Beziehungsebene in Teamarbeiten aussehen? Nennen Sie Beispiele. Überlegen Sie, wie die drei Instanzen in diesen Fällen aussehen. \newline \newline \newline
%\textbf{Aufgabe 4.}\newline Was sind mögliche Quellen für solche Konflikte? \newline \newline
%\textbf{Aufgabe 5.}\newline Wie kann man die Fälle aus 2. und 3. lösen? \newline 
%\newpage


%\textbf{ALTERNATIVE}\newline 

%\textbf{Szenarien:}\newline
%\textbf{Technologieauswahl:} Ein Teammitglied bevorzugt die Verwendung einer bestimmten Programmiersprache oder Technologie, während ein anderes Mitglied eine alternative Technologie bevorzugt. Dies kann zu einem Konflikt führen, da beide Parteien überzeugt sind, dass ihre Wahl die beste für das Projekt ist.\newline \newline

%\textbf{Verteilung von Verantwortlichkeiten:} Ein Teammitglied könnte das Gefühl haben, dass bestimmte Mitglieder die Arbeit nicht gerecht aufteilen oder ihren Anteil an Verantwortlichkeiten nicht übernehmen. Dies kann zu Spannungen führen, insbesondere wenn das Gefühl entsteht, dass einige Teammitglieder die Last auf andere abwälzen.\newline \newline

%\textbf{Kulturelle Unterschiede:} Aufgrund kultureller Unterschiede interpretieren Teammitglieder nonverbale Signale oder Kommunikationsstile möglicherweise unterschiedlich. Dies kann zu Missverständnissen und Konflikten führen, insbesondere wenn es um die Interpretation von Kritik oder Feedback geht.\newline \newline \newline

%\textbf{Aufgabe:}\newline \newline Analysieren Sie diese Szenarien mit dem Eisberg-Modell. Was sind hier die Konfliktpotenziale? Wie sind die Zustände der drei Instanzen in diesen Konflikten? Wie können Konflikte wie solche konkret gelöst werden?
%\newpage

%\textbf{ALTERNATIVE}\newline

%\textbf{Aufgabe:}\newline
%Mögliche Konfliktpotenziale in Gruppenarbeiten sind kulturelle Unterschiede, Sprachbarrieren, unterschiedliche Kenntnisstände sowie unterschiedliche Erwartungen oder Motivationen. Wie können aus solchen Quellen Konflikte entstehen? Geben Sie beispielhaft Szenarien an. Untersuchen Sie diese Szenarien mithilfe des Eisbergmodells und achten Sie dabei auf die Zustände der drei Instanzen. Wie können diese Konflikte gelöst und zukünftig vorgebeugt werden? Nehmen Sie sich hier auch das Modell zur Hilfe.
\newpage

%\textbf{ALTERNATIVE}\newline

\textbf{Aufgabe:}\newline \newline
Szenario: Sie arbeiten in einer Gruppe aus 6 Leuten. 2 dieser Leute haben Programmieren im Laufe eines Ingenieur-Studiums in Deutschland und 2 im Ausland gelernt. Eine Person hat Familie, zwei weitere haben gesundheitliche Einschränkungen (Person A: Angststörung, Person B: eine chronische Lungenerkrankung). 3 der Teammitglieder haben einen Migrationshintergrund, alle sind auf der Stufe B2, einer absolviert aktiv einen weiteren Deutschkurs. Alle Teammitglieder haben unterschiedliche kulturelle Hintergründe oder wurden unter verschiedensten Umständen großgezogen und haben somit auch unterschiedliche Verständnisweisen was Offenheit, Arbeitsethik und Kommunikationsstil betrifft. \newline \newline \textbf{Person A:} verschweigt gerne, dass sie Schwierigkeiten hat und bittet ungerne um Feedback. \newline \textbf{Person B:} Tut sich schwer darin, sich selbst einzuschätzen und übernimmt somit zu viel oder zu wenig der Arbeitslast, wodurch viele Puffer/Umplanungen nötig sind. \newline \textbf{Person C:} Kann Deutsch, drückt sich jedoch teilweise undeutlich aus durch den technischen Jargon, der in die Sätze eingebaut werden muss. \newline \textbf{Person D:} Hat starke Unterschiede im Verständnis der Arbeitsethik und beschäftigt sich mit dem Minimum, stellt somit häufig eine unterbelastete Ressource dar. \newline \textbf{Person E:} Ist ein Platzhirsch, welcher andere Lösungswege nicht anerkennt und andere Beiträge unterbricht. \newline \textbf{Person F:} Hat Familie, darunter 2 Kleinkinder.\newline \newline

Entwickeln Sie zwei Konfliktszenarien basierend auf Kommunikationsfehlern mithilfe der oben gegebenen Informationen über das Team.\newline \newline
Analysieren Sie die entwickelten Szenarien mit dem Eisbergmodell und finden Sie Lösungswege für die aufgekommenen Konflikte.
\end{flushleft}
\end{document}
