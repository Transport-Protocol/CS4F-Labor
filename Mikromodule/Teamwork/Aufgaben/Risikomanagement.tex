\documentclass[a4paper]{article}
\usepackage[utf8]{inputenc}
\usepackage[T1]{fontenc}
\usepackage[ngerman]{babel}


\title{\vspace{-2cm}Risikomanagement}
\author{CS4F}
\date{\today}

\begin{document}

\maketitle

In Projekten gibt es den Verantwortungsbereich des Risikomanagements. Dieser kann entweder eine eigene Rolle bilden oder die Verantwortung des Projektmanagers sein. Risiken innerhalb eines Projekts sind speziell Dinge oder Situationen, die die erfolgreiche Durchführung des Projekts gefährden und müssen frühzeitig erkannt werden \footnote{https://www.bva.bund.de/DE/Services/Behoerden/Beratung/Beratungszentrum/GrossPM/s-o-s\_handbuch/stda\_sos-kap8\_risikomgmt.html}.
Auch die Behandlung der Risiken und Vorbeugung dieser ist Teil des Risikomanagements.\\
\\ Um dies zu ermöglichen muss im Team jedoch ein offenes und kritikfähiges Klima herrschen, sodass Risiken besser identifiziert oder gemeldet werden können. Dazu gehört auch, dass eigene Arbeitsumstände und mögliche Risiken offen dargelegt werden, sodass sie im Management berücksichtigt werden können. Auch die Person, die das Risikomanagement betreibt muss ehrlich mit sich selbst sein und Probleme im eigenen Vorgehen erkennen, damit am Ende jedes Teammitglied sich an der Risikoreduktion beteiligen kann.\\


Wie auch andere Aufgaben und Verantwortlichkeiten sollten Risiken ausgewogen und priorisiert werden, um ein gutes Ressourcenmanagement und somit auch eine erfolgreiche Projektdurchführung zu ermöglichen.\\\\\\
\newpage
\begin{flushleft}
\textbf{Aufgabe:}\newline \newline
Szenario: Sie arbeiten in einer Gruppe aus 6 Leuten. 2 dieser Leute haben Programmieren im Laufe eines Ingenieur-Studiums in Deutschland und 2 im Ausland gelernt. Eine Person hat Familie, zwei weitere haben gesundheitliche Einschränkungen (Person A: Angststörung, Person B: eine chronische Lungenerkrankung). 3 der Teammitglieder haben einen Migrationshintergrund, alle sind auf der Stufe B2, einer absolviert aktiv einen weiteren Deutschkurs. Alle Teammitglieder haben unterschiedliche kulturelle Hintergründe oder wurden unter verschiedensten Umständen großgezogen und haben somit auch unterschiedliche Verständnisweisen was Offenheit, Arbeitsethik und Kommunikationsstil betrifft. \newline \newline \textbf{Person A:} verschweigt gerne, dass sie Schwierigkeiten hat und bittet ungerne um Feedback. \newline \textbf{Person B:} Tut sich schwer darin, sich selbst einzuschätzen und übernimmt somit zu viel oder zu wenig der Arbeitslast, wodurch viele Puffer/Umplanungen nötig sind. \newline \textbf{Person C:} Kann Deutsch, drückt sich jedoch teilweise undeutlich aus durch den technischen Jargon, der in die Sätze eingebaut werden muss. \newline \textbf{Person D:} Hat starke Unterschiede im Verständnis der Arbeitsethik und beschäftigt sich mit dem Minimum, stellt somit häufig eine unterbelastete Ressource dar. \newline \textbf{Person E:} Ist ein Platzhirsch, welcher andere Lösungswege nicht anerkennt und andere Beiträge unterbricht. \newline \textbf{Person F:} Hat Familie, darunter 2 Kleinkinder.\newline \newline


\textbf{Aufgabe 1.}\newline Was für Risiken können Sie in einem Projekt mit dem oben aufgeführten Team erkennen? Berücksichtigen Sie dabei die unterschiedlichen Ansichten und Hintergründe der Mitglieder.\newline

\textbf{Aufgabe 2.}\newline Priorisieren Sie die Risiken. Wie sieht die Liste nun aus? Wonach wurde priorisiert? Begründen Sie ihre Prioritäten.\newline

\textbf{Aufgabe 3.} Wie kann Ihr am höchsten priorisiertes Risiko behandelt und vorgebeugt werden?
\end{flushleft}

% % in case of needing citations
% \bibliographystyle{plainurl}
% \bibliography{refs}

\end{document}
