\documentclass{article}
\usepackage{graphicx} % Required for inserting images

\title{Effiziente Kommunikation}
\author{CS4F}
\date{May 2024}

\begin{document}

\maketitle

%Kommunikation kann effizient sein durch: bessere Übermittlungsqualität d.h. Minimierung von Störungen in En- und Dekodierung sowie Übermittlung.\\
%Die Parteien sollten mit der Kommunikationsart/-Methode vertraut sein. Sie sollten sich angemessen und den Zeilen entsprechend verhalten und reflektieren können, um sich anpassen zu können.\\\\
%(S.108ff.)
%Mögliche Probleme, die eine Kommunikation  ineffizient gestalten: Einseitigkeit, Unkontrolliert in Menge, Inhalt, Präzision, Genauigkeit oder Emotion.
Gute Kommunikation motiviert und unterstützt qualitative Arbeit.\\\\Wenn die Kommunikation innerhalb eines Teams ineffizient gestaltet ist, so ist sie einseitig, unkontrolliert in der Menge,  ungenau und/oder mangelhaft im Inhalt. Durch Vorurteile, Unterbechungen, emotionalen Auseinandersetzungen usw. wird die Ineffizienz gestärkt\footnote{M.Ant, M.C. Nimmerfroh, C. Reinhard, \textit{Effiziente Kommuniaktion}, 1. Wiesbaden: Springer Gabler, 2013. doi: https://doi.org/10.1007/978-3-658-01318-9}. Effiziente Kommunikation hingegen zeichnet sich darin aus, dass die Übermittlung mit wenig Verlust stattfindet, das Verständnis der Nachricht also mit der Intention des Senders übereinstimmt. Die situationsbedingte Meidung möglicher Missverständnisse und Störfaktoren trägt zur Effizienz bei und maximiert den Nutzen. Effiziente Kommunikation sollte zu neuen Sichweisen, Ideen und möglichen Lösungsansätzen zu einem vorher kommunizierten Problem führen.\\

\textit{Wie kann effizient kommuniziert werden?}\\

%\begin{itemize}
 %   \item Konsistenz: Widerspruchsfreie und klare Argumentation
 %   \item Kohärenz: logische Zusammenhänge auch in emotionalen Kommunikationen
  %  \item Kontingenz: es existieren mehrere Wahrheiten, abgängig vom Verständnis der Gesprächsteilnehmer
  %  \item Zielgerichtetheit: zielgerichtete und kontextuelle Kommunikation führt zum Ziel
   % \item Hoher Status: damit wird nicht ein Platzhirsch gemeint, sondern jemand, der interessenunabhängig agieren kann
   % \item Inhaltliche Orientierung: es kann eine rationale, emotionale, moralische oder Plausibilitäts-Argumentation sein, jedoch nicht eine emotionale Auseinandersetzung auf persönlicher Ebene. Die Orientierung hat zur Folge, dass sich auf der selben Kommunikationsebene ausgetauscht wird
   % \item Determination: man sollte sich und seine Position überzeugt kommunizieren
   % \item Autonomie: die Fähigkeit eine eigene Meinung zu entwickeln und diese zu verteidigen oder anzupassen
   % \item Investition: persönliche Opfer (zB Diskussionszeit, Recherche...) für die Meinungsvertretung
   % \item Akzeptanz: andere Standpunkte sollten akzeptiert und empathisch nachvollzogen werden
   % \item Einfachheit, Eindeutigkeit: für daas Verständnis ist es besser, wenn vorerst weniger Argumente oder Inhalte präsentiert und diskutiert werden
   % \item Zuhörfähigkeit: aktives Zuhören hilft beim verstehen und der interpersonellen Beziehungen
   % \item Sprechfähigkeit: die eigene Position zu offenbaren ist wichtig, um anderen das Gefühl zu geben, dass auch sie kommunizieren können
   % \item Orientierung an der Problemlösung: das Problem sollte definiert und analysiert werden. Mehrere Möglichkeiten sollten ausgearbeitet und so eine Lösung umgesetzt werden
   % \item Fairness: Hand-in-Hand mit dem Punkt "`Akzeptanz"', die Argumentationen entgegen der anderen Meinungen sollten sachlich erfolgen
   % \item Meinungsverzicht: trotz der eigenen Meinung sollte man offen bleiben, da andere Meinungen den Lösungsweg bilden könnten
   % \item Offenheit: Zuhören, offen sein, Meinungsverzicht
   % \item Alternativen und Szenarien: es existieren mehrere Lösungswege und diese sollten ermittelt und ausgearbeitet werden, bevor einer gewählt wird
   % \item Denken in Spiralen: eine gute Art, zu Diskutieren, ist, wenn statt Gegenargumenten eine Weiterführung anderer Meinungen (und der daraus folgenden Szenarien) basierend auf der eigenen Meinung gebildet wird, bis klar ist dass dies kein passender Lösungsweg ist (zB Widersprüche oder weiteres)
   % \item Konstruktivismus: neue Gedankenwelten durch das Kommunizieren mehrerer Wahrheiten
   % \item Systematisches Denken: Gesamtheit der Sachlage verstehen, indem KLeinigkeiten und äußere Umstände berücksichtigt werden
   % \item Interdisziplinäres Denken: neue Lösungswege oder Ansichten durch neue Denkweisen
   % \item  Methodisches Vorgehen: Struktur durch Methoden zur Projektdurchführung etc minimieren Fehler/Konflikte/...
   % \item Vorgehen in kleinen Schritten: kleinschrittige Argumentationen und Pläne können einfacher verfolgt/befolgt und geplant werden
   % \item Illustrationen: die Erfassbarkeit der eigenen Stellung kann bildlich unterstützt werden zB Modellierungen bei einem Software-Projekt
   % \item Metakommunikation: Nachhaken, nicht inhaltlich, sondern fürs Verständnis oder auch über die Kommunikationsprobleme und -Methoden reden. "Kommunikation über die Kommunikation"
   % \item Nonverbal: nonverbale Ausdrücke stärken die verbalen Ausdrücke oder geben ihnen eine andere Bedeutung\\

Für eine effiziente Kommunikation sollte eine Problematik erfasst und analysiert werden. Daraufhin folgt die Zielsetzung für die Kommunikation, damit dann gemeinsam Lösungswege entwickelt und bewertet werden können, bevor eine schlussendliche Entscheidung getroffen werden kann. Wenn eine Entscheidung besteht, müssen Handungen innerhalb eines Plans eingeleitet werden, sodass die Pläne in die Stufe der Umsetzung gelangen können. Um dies hinzukriegen müssen Teammitglieder:\\\\
\begin{itemize}
    \item Zeit für die Recherche und Diskussion opfern, um sich selbstständig ein Meinungsbild zu schaffen
    \item Ihre Meinung überzeugt, kohärent, logisch und problemorientert vertreten aber auch offen für andere Lösungswege und Ansichten sein \\\\
    \item In einer Diskussion jemanden haben, der interessenunabhängig den Austausch leiten kann und ihn inhaltlich und quantitativ einfach hält
    \item Sich nicht mit ihrer Meinung zurückhalten aber anderen auch zuhören, damit eine offene, austauschfreudige Atmosphäre jeden zum reden antreibt
    \item Feedback zu Ergebnissen und weiteren beiträgen geben, welches positive sowie negative Aspekte berücksichtigt und sachlich bleibt
    \item Verbal sowie nonverbal sachlich bleiben, besonders Gesten können verbale Ausdrücke stärken
    \item Sich bereit erklären, an Meetings teilzunehmen, um alle auf den gleichen Wissensstand zu bringen 
    \item Mehr mit Ich- statt Du-Botschaften arbeiten\newline
    

So kann ein konstruktiver Austausch erfolgen und es können mehrere Lösungswege ermittelt und weitergeführt werden, bevor sich methodisch die besten Wege herauskristallisieren und sich einer ausgesucht werden kann. In der Diskussion nicht nur die eigenen Standpunkte und Argumente zu kommunizieren, sondern auf den Ansätzen anderer aufzubauen führt zusätzlich weitere Wege ein und kann entweder die eigene Meinung stärken oder entkräften. Die Argumentation ist auch dann zielführend, da sie den Weg bis zum Schluss gelaufen ist und so entweder einen guten oder zu vermeidenden Weg findet.
    

 

\begin{flushleft}
\textbf{Aufgabe:}\newline \newline
Szenario: Sie arbeiten in einer Gruppe aus 6 Leuten. 2 dieser Leute haben Programmieren im Laufe eines Ingenieur-Studiums in Deutschland und 2 im Ausland gelernt. Eine Person hat Familie, zwei weitere haben gesundheitliche Einschränkungen (Person A: Angststörung, Person B: eine chronische Lungenerkrankung). 3 der Teammitglieder haben einen Migrationshintergrund, alle sind auf der Stufe B2, einer absolviert aktiv einen weiteren Deutschkurs. Alle Teammitglieder haben unterschiedliche kulturelle Hintergründe oder wurden unter verschiedensten Umständen großgezogen und haben somit auch unterschiedliche Verständnisweisen was Offenheit, Arbeitsethik und Kommunikationsstil betrifft. \newline \newline \textbf{Person A:} verschweigt gerne, dass sie Schwierigkeiten hat und bittet ungerne um Feedback. \newline \textbf{Person B:} Tut sich schwer darin, sich selbst einzuschätzen und übernimmt somit zu viel oder zu wenig der Arbeitslast, wodurch viele Puffer/Umplanungen nötig sind. \newline \textbf{Person C:} Kann Deutsch, drückt sich jedoch teilweise undeutlich aus durch den technischen Jargon, der in die Sätze eingebaut werden muss. \newline \textbf{Person D:} Hat starke Unterschiede im Verständnis der Arbeitsethik und beschäftigt sich mit dem Minimum, stellt somit häufig eine unterbelastete Ressource dar. \newline \textbf{Person E:} Ist ein Platzhirsch, welcher andere Lösungswege nicht anerkennt und andere Beiträge unterbricht. \newline \textbf{Person F:} Hat Familie, darunter 2 Kleinkinder.
\newline \newline

Entwickeln Sie zwei Konfliktszenarien zu ineffizienter Kommunikation  mithilfe der oben gegebenen Informationen über das Team.\newline \newline
Analysieren Sie die entwickelten Szenarien mit den Informationen zur effizienten/ineffizienten Kommunikation. Was macht diese Szenarien zur ineffizienten Kommunikation? Wie könnte diese sich effizienter gestalten lassen?

\end{flushleft}


\end{itemize}

\end{document}
