\documentclass[a4paper]{article}
\usepackage[utf8]{inputenc}
\usepackage[T1]{fontenc}
\usepackage[ngerman]{babel}

\title{\vspace{-2cm}4-Ohren-Modell}
\author{CS4F}
\date{2024}

\begin{document}

\maketitle


Nach dem 4-Ohren-Modell (oder auch dem "`Kommunikationsquadrat"') von Friedemann Schulz von Thun, haben Nachrichten vier Verständnisebenen, die herausgehört werden können\footnote{https://www.schulz-von-thun.de/die-modelle/das-kommunikationsquadrat}
.\\\\ Dazu gehört das "`Sach-Ohr"' (oder "`Sachinformation"'), bei dem es um Sachverhalte und ihren Ausdruck geht. Dabei spielen Wahrheitswerte, Relevanz sowie Vollständigkeit der dargebotenen Fakten eine Rolle.\\\\
Eine weitere Seite des Quadrats ist die "`Selbstkundgabe"'. Dabei werden über den Inhalt hinaus persönliche Bezüge dargelgt wie die eigenen Werte oder Gefühle. Demnach kann über diese Ebene eine persönliche Einschätzung erfolgen.\\\\
Es besteht eine schmale Linie zur "`Beziehungsseite"'. Im Gegensatz zur Selbstkundgabe geht es hier aber nicht nur um das selbst, sondern um die Beziehung und Meinung zur anderen Person, wobei diese Informationen durch non-verbale Kommunikation oder Nachrichtenformulierungen übermittelt werden.\\\\
Zuletzt gibt es noch die "`Appellseite"'. Hier geht es darum, was die Empfängerseite tun muss, könnte oder sollte. Also werden hier Wünsche, Befehle oder Hinweise geäußert.\\\\
Wichtig zu verstehen ist, dass eine Nachricht auf allen vier Ebenen eine Bedeutung hat. Ein bekanntes Beispiel ist das Szenario "`Was ist das Grüne in der Soße?"', in welchem folgende Dinge ausgedrückt werden können:\\


\begin{itemize}
    \item Sach-Ohr: Da ist etwas Grünes in der Soße.
    \item Selbstkundgabe: Ich weiß nicht, was das ist.
    \item Beziehung: Du weißt bestimmt, was das ist.
    \item Appell: Sag mir, was das Grüne in der Soße ist.\\
    
\end{itemize}
Auf der Seite der Empfänger kann diese Nachricht auch anders auf jeder Ebene aufgegriffen werden:
\newpage
\begin{itemize}
    \item Sach-Ohr: Da ist etwas Grünes in der Soße.
    \item Selbstkundgabe: Ich mag das Essen nicht.
    \item Beziehung: Du kannst nicht kochen.
    \item Appell: Mach' die Soße nächstes mal anders.\\\\\\
    
\end{itemize}

\begin{flushleft}

\textbf{Aufgabe:}\newline \newline
Szenario: Sie arbeiten in einer Gruppe aus 6 Leuten. 2 dieser Leute haben Programmieren im Laufe eines Ingenieur-Studiums in Deutschland und 2 im Ausland gelernt. Eine Person hat Familie, zwei weitere haben gesundheitliche Einschränkungen (Person A: Angststörung, Person B: eine chronische Lungenerkrankung). 3 der Teammitglieder haben einen Migrationshintergrund, alle sind auf der Stufe B2, einer absolviert aktiv einen weiteren Deutschkurs. Alle Teammitglieder haben unterschiedliche kulturelle Hintergründe oder wurden unter verschiedensten Umständen großgezogen und haben somit auch unterschiedliche Verständnisweisen was Offenheit, Arbeitsethik und Kommunikationsstil betrifft. \newline \newline \textbf{Person A:} verschweigt gerne, dass sie Schwierigkeiten hat und bittet ungerne um Feedback. \newline \textbf{Person B:} Tut sich schwer darin, sich selbst einzuschätzen und übernimmt somit zu viel oder zu wenig der Arbeitslast, wodurch viele Puffer/Umplanungen nötig sind. \newline \textbf{Person C:} Kann Deutsch, drückt sich jedoch teilweise undeutlich aus durch den technischen Jargon, der in die Sätze eingebaut werden muss. \newline \textbf{Person D:} Hat starke Unterschiede im Verständnis der Arbeitsethik und beschäftigt sich mit dem Minimum, stellt somit häufig eine unterbelastete Ressource dar. \newline \textbf{Person E:} Ist ein Platzhirsch, welcher andere Lösungswege nicht anerkennt und andere Beiträge unterbricht. \newline \textbf{Person F:} Hat Familie, darunter 2 Kleinkinder.\newline \newline


\textbf{Aufgabe 1.}\newline Was können im Bezug zum Kommunikationsmodell Probleme in der oben aufgeführten Gruppenarbeit sein? Erstellen Sie zwei beispielhafte Szenarien in Ihrer Gruppe.\newline

\textbf{Aufgabe 2.}\newline Auf welcher Ebene entstehen die von Ihnen ausgewählten Probleme? Auf welcher Ebene können diese aufgehoben werden? Wie würdet ihr im Konfliktmanagement mit den ausgedachten Situationen umgehen?\newline


\end{flushleft}
\end{document}
